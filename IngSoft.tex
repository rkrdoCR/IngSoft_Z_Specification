\documentclass[12pt,a4paper,table]{article}
\usepackage{zed-csp}
\usepackage{graphicx}
\graphicspath{{images/}}
\usepackage[utf8]{inputenc}
\usepackage[spanish]{babel}
\usepackage{float}
\usepackage{fancyhdr}
\usepackage{tikz}
\usetikzlibrary{shapes,arrows}
\usetikzlibrary{matrix}

\usepackage{multirow}
\usepackage{hhline}
\usepackage{float}
\usepackage{array, makecell}
\usepackage{xcolor}
\setcellgapes{12pt}

\tikzstyle{line} = [draw, -latex']

\usetikzlibrary{arrows.meta,
                calc, chains,
                quotes,
                positioning,
                shapes.geometric}               

\usepackage[nottoc]{tocbibind} %Adds "References" to the table of contents
\usepackage{url}

%% Comandos
\renewcommand{\familydefault}{\rmdefault}
\renewcommand*{\defs}{\mathrel{\widehat{=}}}

\title{Entrega de paquetes. Bodegaje y entrega al comprador.\\Especificación en Z.}
%%

\pagestyle{fancy}
\fancyhead[LE,RO]{Entrega de paquetes. Bodegaje y entrega al comprador.}
\fancyhead[RE,LO]{}

\newcommand{\myBox}[3]{%	
	\begin{tcolorbox}[colback=#1,colframe=#2]
		\centering{#3}
	\end{tcolorbox}
}
 
%
\begin{document}


\begin{titlepage}
	\centering
	\includegraphics[width=0.33\textwidth]{Teclogocompleto.jpg}\par\vspace{1cm}
	{\scshape\large \textbf{Instituto Tecnológico de Costa Rica }\par}
	\vspace{1cm}
	{\scshape\Large MC 7204 Ingeniería de Software\par}
	\vspace{1.5cm}
	{\Large\bfseries 
	Entrega de paquetes.\\Bodegaje y entrega al comprador.\\Especificación en Z.\par}
	\vspace{2cm}
	{\Large\itshape Marco Acuña  \\
	 Ricardo Alfaro\par}
	\vfill
	Profesor:\par
	Ignacio Trejos Zelaya\textsc{}
	\vfill

% Bottom of the page
	{\large} Noviembre 2020 \par
\end{titlepage}

\tableofcontents
\newpage


\section{Resumen}
El comercio ha sido una actividad fundamental en la historia de la humanidad y como tal ha evolucionado de acuerdo a las tecnologías disponibles en las diferentes épocas. 
En el mundo actual se pueden observar fácilmente dos ejes principales en el establecimiento de las relaciones comerciales: 
\begin{enumerate}
\item La comodidad y seguridad que brindan los medios electrónicos de pago para adquirir productos en el exterior a través de sitios web de tipo comercio electrónico (e-commerce)
\item El traslado de los bienes adquiridos que realizan las empresas de logística y mensajería apoyadas también en los beneficios tecnológicos actuales, tanto a nivel de transporte como de información.
\end{enumerate}
Presentamos un trabajo de especificación formal enfocado en modelar parte del proceso que contempla el eje mencionado anteriormente en el punto número 2. En la siguiente lista se resume un flujo común (hoy en día) de adquisición y entrega de un paquete:
\begin{enumerate}
\item La persona que desea adquirir un producto cuenta con lo siguiente:
\begin{enumerate}
\item Una cuenta con alguna empresa de logística y mensajería a nivel nacional.
\item Algún medio de pago electrónico que le permita hacer compras en el extranjero.
\end{enumerate}
\item El comprador adquiere uno o varios productos en un sitio de comercio electrónico.
\item El comprador envía los productos adquiridos al casillero que le fue asignado por parte de la empresa de logística y mensajería.
\item La empresa de logística y mensajería procesa el paquete recibido y lo asigna a una carga y a un vuelo que lo traerá al país.
\item Una vez en el país el paquete es procesado por aduanas para lo referente a tramites de nacionalización.
\item La empresa de logística y mensajería mueve el paquete a sus bodegas en el país para proseguir con lo referente a la entrega del paquete al comprador.
\item El flujo finaliza cuando el comprador ha recibido el paquete.
\end{enumerate}

\section{Definición del problema}
La entrega pronta y precisa de los paquetes adquiridos mediante compras en el exterior. Se toma en cuenta la parte del proceso que inicia luego de que el paquete ha sido procesado por la aduana (tramites de nacionalización).
\subsection{Bodegaje y entrega al comprador.}
Para efectos de este trabajo se denomina dicho subproceso como \textit{Bodegaje y entrega al comprador} y los pasos que comprende son:
\begin{enumerate}
\item El paquete existe en la bodega de la empresa de mensajería y no está asignado a alguna ruta para su entrega al comprador.
\item El paquete es retirado de la bodega, es asignado a un mensajero y puesto en ruta.
\item El mensajero intenta la entrega del paquete, acá se pueden presentar los siguientes escenarios:
\begin{enumerate}
\item El comprador está disponible en la dirección de entrega y recibe el paquete.
\item El comprador no se encuentra en la dirección de entrega, por lo que el paquete debe ser devuelto de nuevo a la bodega.
\end{enumerate}
\item Si el paquete no fue recibido por el comprador, el mismo debe volver a ser ingresado a la bodega, lo cual permite que el paquete esté disponible de nuevo para reiniciar el proceso de entrega.
\end{enumerate}

Para cada etapa del proceso descrita en este apartado, el sistema genera un registro en el historial de rastreo del paquete. Este registro es simplemente una entrada de datos asociada a un tipo de movimiento que refleja el estado actual del paquete. El manejo del histórico de movimientos del paquete no está incluido dentro del alcance de este trabajo, sin embargo, se incluye la figura 1 con el objetivo de brindar una intuición de la posible estructura de un registro de movimiento en el historial. 

\tikzset{ 
    table/.style={
        matrix of nodes,
        row sep=-\pgflinewidth,
        column sep=-\pgflinewidth,
        nodes={
            rectangle,
            draw=black,
            align=center
        },
        minimum height=1.5em,
        text depth=0.5ex,
        text height=2ex,
        nodes in empty cells,
%%
        every even row/.style={
            nodes={fill=gray!20}
        },
        column 1/.style={
            nodes={text width=2em,font=\bfseries}
        },
        row 1/.style={
            nodes={
                fill=green!30,
                text=black,
                font=\bfseries
            }
        }
    }
}

\begin{figure}[h]
\centering
\begin{tikzpicture}
\matrix (first) [table,text width=6em]
{
  Id  & IdPaq & FechaHora  &  TipoMov  & IdEmp & Notas\\
  1   & 10023 & dd/mm/yyyy &  enRuta   & E012  & en ruta \\
};
\end{tikzpicture}
\caption{\textit{Registro de evento.}} \label{fig:M1}
\end{figure}

\subsection{Diagrama de flujo del subproceso.}
El siguiente diagrama de flujo ilustra la parte del proceso que será formalizada en este trabajo, los registros del historial son generados en los procesos.

\begin{figure}[h]
\centering
\begin{tikzpicture}[
    node distance=1.69cm,
    startstop/.style={rectangle, rounded corners, minimum width=3cm, minimum height=1cm,text centered, draw=black, fill=red!30},
    process/.style={rectangle, minimum width=3cm, minimum height=1cm, text centered, draw=black, fill=orange!30},
    io/.style={trapezium, trapezium left angle=70, trapezium right angle=110, minimum width=3cm, minimum height=1cm, text centered, draw=black, fill=blue!30},
    decision/.style={diamond, minimum width=3cm, minimum height=1cm, text centered, draw=black, fill=green!30},
    ]

    \node (node0) [startstop]                             					{Inicio};
    \node (node1) [process, align=center, below of=node0]               	{Paquete esta\\en bodega};
    \node (node2) [decision, align=center, below of=node1,yshift=-1cm]		{Asignable?};     
    \node (node3) [process, align=center, below of=node2, yshift=-1cm] 		{Asignar ruta,\\mensajero y enviar};
    \node (node4) [process, align=center, right of=node2, xshift=4.5cm]   	{Mantener\\en bodega};
    \node (node5) [decision, align=center, below of=node3, yshift=-1cm]		{Entregado?};
    \node (node6) [process, align=center, below of=node5, yshift=-1cm]		{Registrar\\entrega};
    \node (node7) [process, align=center, right of=node5, xshift=2cm]   	{Devolver\\a bodega};
    \node (node8) [startstop, below of=node6, xshift=3.7cm]					{Fin};

    \draw [arrows=-Stealth] (node0) --node[anchor=east]             {}		(node1);
    \draw [arrows=-Stealth] (node1) --node[anchor=east]             {}		(node2);
    \draw [arrows=-Stealth] (node2) --node[anchor=east]             {si}	(node3);
    \draw [arrows=-Stealth] (node2) --node[anchor=south]            {no}    (node4);
    \draw [arrows=-Stealth] (node3) --node[anchor=south]            {}		(node5);
    \draw [arrows=-Stealth] (node5) --node[anchor=east]             {si}	(node6);
    \draw [arrows=-Stealth] (node5) --node[anchor=south]            {no}    (node7);
    \path [line, arrows=-Stealth] (node4) |- (node8);
    \path [line, arrows=-Stealth] (node7) -- (node8);    
    \path [line, arrows=-Stealth] (node6) |- (node8);

  \end{tikzpicture}
\caption{\textit{Flujo. Bodegaje y entrega al comprador}} \label{fig:M1}
\end{figure}


\newpage
\section{Especificación en Z}

Se presenta una especificación formal en Z para el subproceso denominado en este trabajo como \textit{Bodegaje y entrega al comprador} el cual es parte de un sistema de manejo y entrega de paquetes cuya finalidad es de que sea utilizado por empresas de servicios logística y de mensajería.  

\subsection{Variables globales}
Se definen los tipos \textit{Paquete}, \textit{Cliente} y \textit{Empleado}  como los conjuntos(tipos) que se utilizan en esta especificación. 

\begin{zed}
[Paquete,~Cliente,~Empleado]
\end{zed}

\subsection{Respuestas del sistema}

Cada operación del sistema debe reportar su resultado mediante un mensaje de respuesta que informe al usuario acerca del estado actual del proceso. En esta sección se especifican las respuestas utilizadas en métodos que no presentan condiciones de falla y que, por lo tanto, su salida siempre es la misma. Otras operaciones, por el contrario, son definidas como robustas más adelante en esta especificación.

\begin{zed}
Respuesta~~::=~~paqueteAlmacenado\\
~~~~~~|~~consultaPaqueteOK\\
\end{zed}

\subsection{Mensajeros. Disponibilidad y asignación.}
\indent Los empleados son catalogados por tipos. Para cumplir con el objetivo de que los paquetes lleguen a sus dueños es fundamental la participación de los mensajeros, que son el único tipo de empleado participante del proceso que se modela en este trabajo. Asignar exitosamente un paquete a un mensajero implica el cumplimiento de ciertas precondiciones por lo que es necesario modelar los posibles escenarios en los que la asignación no se puede realizar.

\subsubsection{Tipo Mensajero.}
Las empresas de logística y mensajería cuentan con diferentes roles que le son asignados a sus empleados de acuerdo al puesto. Estos roles son modelados como etiquetas y para efectos del subproceso a formalizar solo se necesita especificar el rol de tipo mensajero.

\begin{zed}
Rol~~::=~~TMensajero
\end{zed}

La clase mensajero representa de manera única a este tipo de empleados y se define a continuación.

\begin{schema}{Mensajero}
mensajero : Empleado\\
tipo : Rol
\where
tipo = TMensajero
\end{schema}

\subsubsection{Disponibilidad y asignación.}
Como se menciona al inicio de esta sección es importante cubrir las posibles salidas del proceso de asignación, con el fin de garantizar que el sistema hace un manejo correcto de la disponibilidad y asignación de los mensajeros se presenta la siguiente definición libre de tipos:

\begin{zed}
ASIGNA~~::=~~asignaExitoso\\
~~~~~~|~~paqueteNoEstaEnBodega\\
~~~~~~|~~paqueteYaEstaAsignado\\
~~~~~~|~~mensajeroNoDisponible\\
\end{zed}

Esta definición permite crear esquemas que, en conjunción con estas respuestas y entre ellos, brindan un mecanismo robusto que da solución a la tarea de presentar adecuadamente los posibles resultados de la acción de asignar paquetes a un mensajero. Dichos posibles resultados son:
\begin{itemize}
\item Se cumplen las precondiciones y es posible asignar el paquete.
\item Se intenta procesar un paquete que no está en la bodega.
\item Se intenta procesar un paquete que ya está asignado.
\item El mensajero ya no puede atender más entregas de paquetes.
\end{itemize} 

La definición de estos esquemas se presenta más adelante en este documento en la sección 3.6 que es donde se definen las operaciones.

\subsection{Bodegaje y entrega al comprador}

En la fase de bodegaje y entrega al comprador los paquetes inicialmente se encuentran en la bodega, el objetivo final de la empresa de logística y mensajería es entregar el paquete al comprador, para ello debe de fijar una ruta y asignar el paquete a un mensajero, los paquetes que aún están en la bodega y los que ya fueron puestos en ruta están pendientes de entregar a la persona (comprador) a la cual pertenecen.

%\indent Previo a intentar la entrega del paquete, el mensajero debe comprobar la disponibilidad del comprador para recibirlo en la dirección de entrega, para dicho efecto se provee las siguientes etiquetas.
%
%\begin{zed}
%Disponible~~::=~~Si~~|~~No
%\end{zed}
%
%Estas etiquetas le permiten al mensajero marcar la entrega del paquete como exitosa o fallida y se 

%Estas etiquetas extienden la definición de un comprador agregándole un atributo que permite determinar la disponibilidad del mismo para recibir el paquete.

%\begin{schema}{Comprador}
%comprador : Cliente\\
%disponible : Disponible
%\where
%(disponible = Si \lor disponible = No)
%\end{schema}

La clase principal del sistema se presenta a continuación.

\begin{schema}{EntregaPaquete}
enBodega, enRuta, pendienteEntrega, entregado : \power Paquete\\
comprador : \power Cliente\\ 
mensajero : \power Mensajero\\
pertenece : Paquete \pfun Cliente\\
asignado : Paquete \pfun Mensajero
\where
\disjoint \langle pendienteEntrega, entregado \rangle\\
\langle enBodega, enRuta\rangle \partition pendienteEntrega \\
pendienteEntrega = \dom pertenece \\
enRuta = \dom asignado \\
\ran asignado \subseteq mensajero \\
\ran pertenece \subseteq comprador
\end{schema}

Los conjuntos \textit{pendienteEntrega} y \textit{entregado} son disjuntos (su intersección es siempre es vacía) mientras que \textit{pendienteEntrega} es la union de lo que esta \textit{enBodega} con lo que esta \textit{enRuta} y esto representa el dominio de la funcion \textit{pertenece}. Los \textit{mensajeros} y los \textit{compradores} estan dados por el rango de las funciones \textit{asignado} y \textit{pertenece} respectivamente.

\subsection{Estado inicial}

Inicialmente los paquetes se encuentran en la bodega, no han sido asignados a un mensajero por lo tanto no están asociados a ninguna ruta. El total de paquetes por entregar es igual al de los paquetes que se encuentran en la bodega y el total de entregados corresponde a 0. El comprador está definido en el rango de la función pertenece, el conjunto de mensajeros es inicializado mediante una variable de entrada.

\begin{schema}{InitEntregaPaquete}
EntregaPaquete'\\
mensajero? : \power Mensajero\\
pertenece? : Paquete \pfun Cliente
\where
pertenece' = pertenece?\\
mensajero' = mensajero?\\
enBodega' = \dom pertenece'\\
enRuta' = \emptyset\\
asignado' = \emptyset\\
entregado' = \emptyset\\
comprador' = \ran pertenece'\\
pendienteEntrega' = enBodega'
\end{schema}

El siguiente diagrama muestra el estado del sistema en cuanto a la ubicación de los paquetes una vez inicializado el sistema. Los puntos en café son paquetes que aún no han ingresado a la bodega.

\begin{figure}[H]
\centering
\begin{tikzpicture}

    \node at (6.72,4.75) {Paquete};

    \draw  (-2.5,-1.5)  rectangle (7.5,4.5);

    \draw[fill=gray!30] (0.7,1.5) ellipse (3cm and 2.5cm);
    \draw[fill=white] (1.8,1) ellipse (0.75cm and 1.25cm);
    \draw[fill=white] (-0.3,1) ellipse (1cm and 1.5cm);
    
    \draw[] (5.7,1.5) ellipse (1.5cm and 2cm);    
    
    \node at (5.7,2.5) {entregado};
    \node at (0.5,3) {pendienteEntrega};
    \node at (-0.27,1.3) {enBodega};
    \node at (1.8,1.5) {enRuta};
    
    \draw [fill] (0.2,2) circle [radius=1.5pt];
    \draw [fill] (-1,0.75) circle [radius=1.5pt];
    \draw [fill] (-0.75,0.5) circle [radius=1.5pt];
    \draw [fill] (-0.5,1) circle [radius=1.5pt];
    \draw [fill] (-0.5,-0.15) circle [radius=1.5pt];
    \draw [fill] (0,0.5) circle [radius=1.5pt];

	\draw [fill, brown] (2,4.2) circle [radius=1.5pt];
    \draw [fill, brown] (-2,3.8) circle [radius=1.5pt];
    \draw [fill, brown] (-2,-1.2) circle [radius=1.5pt];
    \draw [fill, brown] (-1.5,-1) circle [radius=1.5pt];
    \draw [fill, brown] (-1.5,3.5) circle [radius=1.5pt];    
    
\end{tikzpicture}
\caption{\textit{Estado inicial}} \label{fig:M1}
\end{figure}

\subsection{Operaciones}

En esta sección se provee la especificación formal de las operaciones que se implementaran en el módulo de \textit{Bodegaje y entrega al comprador}. Para las operaciones que lo permiten, se presenta una tabla en la que se condensa la información referente a las entradas, estados y salidas de las mismas. Finalmente se introducen también algunos diagramas de Venn con el fin de que permitan apreciar con mayor claridad del estado de los paquetes luego de que una o varias operaciones son ejecutadas.

\subsubsection{Consultar paquete}

Mediante esta operación se comprueba que existen paquetes pendientes de entrega en la bodega y que los paquetes requeridos se encuentran actualmente almacenados. La respuesta de esta operación está garantizada de ser siempre la misma bajo el supuesto de que siempre hay paquetes en la bodega.

\begin{table}[H]
\center
\makegapedcells
\begin{tabular}{||l|l|l|l||}
\hline
\rowcolor[gray]{.8} \textbf{Entrada} & \textbf{Caso} & \textbf{Estado} & \textbf{Salida(Tipo : Respuesta)} \\
\hline
\hline
\multirow{2}{*} {Paquete} & Exitoso & enBodega & consultaPaqueteOK \\
\cline{2-4} & Exitoso & indefinido & consultaPaqueteOK \\
\hline 
\end{tabular}
\caption{\textit{Consultar Paquete.}} \label{fig:M1}
\end{table}

Si el paquete aún no ha ingresado a la bodega el estado es \textit{indefinido} ya que no se puede determinar en el alcance del subproceso que se está modelando. Al ejecutarse la consulta el resultado de la operación es igualmente considerado como exitoso, la salida es de tipo \textit{Respuesta}.\\
El esquema que representa a esta operación es:

\begin{schema}{ConsultarPaquete}
\Xi EntregaPaquete\\
p? : Paquete\\
r! : Respuesta
\where
%enBodega \neq \emptyset\\
%p? \in enBodega\\
r! = consultaPaqueteOK
\end{schema}

La respuesta \textit{consultaPaqueteOK} es de manejo interno en el sistema e indica que la operación se ejecutó. Para el usuario la respuesta recibida sera un registro con los datos consultados si el paquete está en la bodega o bien un registro en blanco si el paquete no se encuentra, de esta manera se indica que la búsqueda ocurrió y finalizo de forma exitosa, aunque el paquete objetivo no fuese encontrado.

\subsubsection{Desalmacenar paquete y asignar a ruta.}

Una vez que los paquetes son tomados de la bodega estos deben ser asignados al mensajero. Los paquetes pasan a estar en ruta y el mensajero es parte del rango de la función \textit{asignado}.

\begin{table}[H]
\center
\makegapedcells
\begin{tabular}{||l|l|l|l||}
\hline
\textbf{Entradas} & \textbf{Caso} & \textbf{Estado} & \textbf{Salida(Tipo : ASIGNA)} \\
\hline
\hline
\multirow{2}{*} {\vtop{\hbox{\strut Paquete}\hbox{\strut Mensajero}}} & Exitoso & enRuta & asignaExitoso\\
\cline{2-4} & PqtNoDisponible & indefinido & paqueteNoEstaEnBodega \\
\cline{2-4} & PqtYaAsignado & enRuta & paqueteYaEstaAsignado \\
\cline{2-4} & MsjNoDisponible & enBodega & mensajeroNoDisponible \\
\hline 
\end{tabular}
\caption{\textit{Desalmacenar paquete y asignar a ruta.}} \label{fig:M1}
\end{table}

%La operación es considerada satisfactoria cuando se cumplen todas las precondiciones y fallida en el caso contrario (cuando alguna precondición no se cumple).

\subsubsection*{Salidas de la operación. Esquemas.}

\begin{enumerate}
\item La operación concluye satisfactoriamente.

\begin{schema}{DesalmacenarAsignarRutaOK}
\Delta EntregaPaquete\\
p? : Paquete\\
m? : Mensajero\\
r! : ASIGNA
\where
p? \in enBodega\\
p? \notin  enRuta\\
m? \in mensajero \setminus \ran asignado\\
enBodega' = enBodega \setminus \{p?\}\\
enRuta' = enRuta \cup \{p?\}\\
asignado' = asignado \cup \{p? \mapsto m?\}\\
mensajero' = mensajero\\
comprador' = comprador\\
pertenece' = pertenece\\
entregado' = entregado\\
pendienteEntrega' = pendienteEntrega
\end{schema}

\subsubsection*{Respuesta al caso exitoso.}
\indent Cuando la asignación concluye con éxito se utiliza el siguiente esquema para reportarlo.

\begin{schema}{AsignaExitoso}
resultado! : ASIGNA
\where
resultado! = asignaExitoso
\end{schema}

Este esquema es utilizado en conjunción con \textit{DesalmacenarAsignarRutaOK} de la siguiente forma:\\

~~~~~~~~$DesalmacenarAsignarRutaOK \wedge AsignaExitoso$\\

Luego de ejecutar la operación \textit{DesalmacenarAsignarRutaOK} la clase \textit{EntregaPaquete} ha sido modificada por el movimiento de paquetes, el estado del sistema se puede apreciar en el siguiente diagrama.

\begin{figure}[H]
\centering
\begin{tikzpicture}

    \node at (6.72,4.75) {Paquete};

    \draw  (-2.5,-1.5)  rectangle (7.5,4.5);

    \draw[fill=gray!30] (0.7,1.5) ellipse (3cm and 2.5cm);
    \draw[fill=white] (1.8,1) ellipse (0.75cm and 1.25cm);
    \draw[fill=white] (-0.3,1) ellipse (1cm and 1.5cm);
    
    \draw[] (5.7,1.5) ellipse (1.5cm and 2cm);    
    
    \node at (5.7,2.5) {entregado};
    \node at (0.5,3) {pendienteEntrega};
    \node at (-0.27,1.3) {enBodega};
    \node at (1.8,1.5) {enRuta};
    
    \draw [fill] (0.2,2) circle [radius=1.5pt];
    \draw [fill] (-1,0.75) circle [radius=1.5pt];
    \draw [fill] (-0.75,0.5) circle [radius=1.5pt];
    \draw [fill] (-0.5,1) circle [radius=1.5pt];
    \draw [fill] (-0.5,-0.15) circle [radius=1.5pt];

	\draw [fill, brown] (2,4.2) circle [radius=1.5pt];
    \draw [fill, brown] (-2,3.8) circle [radius=1.5pt];
    \draw [fill, brown] (-2,-1.2) circle [radius=1.5pt];
    \draw [fill, brown] (-1.5,-1) circle [radius=1.5pt];
    \draw [fill, brown] (-1.5,3.5) circle [radius=1.5pt];       
    
    \draw [fill, blue] (2,0.5) circle [radius=1.5pt]; 
    \draw[{Circle[red]}-Latex] (0,0.5) -- (1.9,0.5);
    
\end{tikzpicture}
\caption{\textit{Desalmacenaje y asignación a ruta}} \label{fig:M1}
\end{figure}

Dentro del universo de paquetes los puntos color café representan los paquetes que aún no han sido ingresados a la bodega. Los puntos de color negro son los paquetes que están actualmente en la bodega. La acción realizada por la operación \textit{DesalmacenarAsignarRutaOK} está representada por el punto de color rojo y el movimiento que lo convierte en un paquete en ruta indicado por la flecha y el punto de color azul.

\item La operación falla al incumplir alguna precondición. Se identifican tres posibles causas de falla para esta operación.

\begin{enumerate}
\item El paquete aún no está en bodega.
\begin{schema}{DesalmacenarAsignarRutaPqtNoDisponible}
\Xi EntregaPaquete\\
p? : Paquete\\
m? : Mensajero\\
r! : ASIGNA
\where
p? \notin enBodega\\
r! = paqueteNoEstaEnBodega
\end{schema}

\item El paquete ya está asignado.
\begin{schema}{DesalmacenarAsignarRutaPqtYaAsignado}
\Xi EntregaPaquete\\
p? : Paquete\\
m? : Mensajero\\
r! : ASIGNA
\where
p? \in  \dom asignado\\
r! = paqueteYaEstaAsignado
\end{schema}

\item El mensajero ya no puede atender más asignaciones de paquetes.
\begin{schema}{DesalmacenarAsignarRutaMsjNoDisponible}
\Xi EntregaPaquete\\
p? : Paquete\\
m? : Mensajero\\
r! : ASIGNA
\where
m? \notin mensajero \setminus \ran asignado\\
r! = mensajeroNoDisponible
\end{schema}

\end{enumerate}
\end{enumerate}

En este caso la salida de la operación es una respuesta de tipo \textit{ASIGNA} que es acorde al escenario que se presenta en cada esquema. El estado del sistema es igual al estado inicial (ver Figura 3) ya que la operación no concluyo satisfactoriamente. Preservar este estado es importante ya que esto permite que el paquete esté disponible de nuevo para ser asignado y puesto en ruta en intentos subsiguientes.

\subsubsection*{Operacion Robusta.}

\[ RDesalmacenarAsignarRuta \defs\\ 
\t1 ~~~~(DesalmacenarAsignarRutaOK \wedge AsignaExitoso)\\ 
\t1 \lor DesalmacenarAsignarRutaPqtNoDisponible\\
\t1 \lor DesalmacenarAsignarRutaPqtYaAsignado\\
\t1 \lor DesalmacenarAsignarRutaMsjNoDisponible\]

La operación \textit{RDesalmacenarAsignarRuta} garantiza ser robusta ya que se combina con la especificación \textit{ASIGNA} y cubre todas las posibles salidas de esta operación.

\subsubsection{Entrega del paquete al comprador.}
Una vez que el paquete ha sido asignado a un mensajero y por ende puesto en ruta el mensajero debe intentar hacer la entrega al comprador.

\begin{table}[H]
\center
\makegapedcells
\begin{tabular}{||l|l|l|l||}
\hline
\textbf{Entradas} & \textbf{Caso} & \textbf{Estado} & \textbf{Salida(Tipo : ENTREGA)} \\
\hline
\hline
\multirow{2}{*} {\vtop{\hbox{\strut Paquete}\hbox{\strut Mensajero}\hbox{\strut Cliente}\hbox{\strut Disponible}}} & Exitoso & entregado & entregaOK \\
\cline{2-4} & Fallido & enRuta & entregaFallida \\
\hline 
\end{tabular}
\caption{\textit{Entrega del paquete al comprador.}} \label{fig:M1}
\end{table} 

Previo a intentar la entrega del paquete, el mensajero debe comprobar la disponibilidad del comprador para recibirlo en la dirección de entrega, para dicho efecto se provee las siguientes etiquetas.

\begin{zed}
Disponible~~::=~~Si~~|~~No
\end{zed}

%Estas etiquetas le permiten al mensajero marcar la entrega del paquete como exitosa o fallida y se 

Los dos posibles escenarios que el mensajero puede enfrentar son que el comprador no se encuentre disponible para la entrega en la dirección especificada o que el comprador si este disponible para la entrega.

\begin{zed}
ENTREGA~~::=~~entregaOK\\
~~~~~~|~~entregaFallida
\end{zed}

El modelado de estas situaciones se detalla a continuación:

\begin{enumerate}
\item El comprador se encuentra en la dirección de entrega y recibe el paquete.

\begin{schema}{EntregaPaqueteOK}
\Delta EntregaPaquete\\
p? : Paquete\\
m? : Mensajero\\
c? : Cliente\\
d? : Disponible\\
r! : ENTREGA
\where
d? = Si\\
entregado' = entregado \cup \{p?\}\\
enRuta' = enRuta \setminus \{p?\}\\
asignado' = asignado \setminus \{p? \mapsto m?\}\\
pertenece' = pertenece \setminus \{p? \mapsto c?\}\\
enBodega' = enBodega\\
comprador' = comprador\\
mensajero' = mensajero
\end{schema}
 
Cuando la asignación concluye con éxito se utiliza el siguiente esquema para reportarlo.

\begin{schema}{EntregaCompletada}
resultado! : ENTREGA
\where
resultado! = entregaOK
\end{schema}

Este esquema es utilizado en conjunción con \textit{DesalmacenarAsignarRutaOK} de la siguiente forma:\\

~~~~~~~~$EntregaPaqueteOK \wedge EntregaCompletada$\\

\item El paquete no pudo ser entregado al comprador. El comprador no está disponible para la recepción del paquete.

\begin{schema}{EtregarPaqueteFallido}
\Xi EntregaPaquete\\
p? : Paquete\\
m? : Mensajero\\
c? : Cliente\\
d? : Disponible\\
r! : ENTREGA
\where
d? = No
\end{schema}

\end{enumerate}

\subsubsection*{Operación Robusta}
Se define una operación robusta que cubre tanto el caso en el que el paquete se entregó con éxito, así como, el caso en el que el paquete no pudo ser entregado al comprador.

\[ REntregaPaquete \defs\\ 
\t1 (EntregaPaqueteOK \wedge EntregaCompletada)
\lor EntregaPaqueteFallido \]

En el siguiente diagrama se puede apreciar el estado del sistema luego de que la entrega se completó de manera satisfactoria.

\begin{figure}[H]
\centering
\begin{tikzpicture}

    \node at (6.72,4.75) {Paquete};

    \draw  (-2.5,-1.5)  rectangle (7.5,4.5);

    \draw[fill=gray!30] (0.7,1.5) ellipse (3cm and 2.5cm);
    \draw[fill=white] (1.8,1) ellipse (0.75cm and 1.25cm);
    \draw[fill=white] (-0.3,1) ellipse (1cm and 1.5cm);
    
    \draw[] (5.7,1.5) ellipse (1.5cm and 2cm);    
    
    \node at (5.7,2.5) {entregado};
    \node at (0.5,3) {pendienteEntrega};
    \node at (-0.27,1.3) {enBodega};
    \node at (1.8,1.5) {enRuta};
    
    \draw [fill] (0.2,2) circle [radius=1.5pt];
    \draw [fill] (-1,0.75) circle [radius=1.5pt];
    \draw [fill] (-0.75,0.5) circle [radius=1.5pt];
    \draw [fill] (-0.5,1) circle [radius=1.5pt];
    \draw [fill] (-0.5,-0.15) circle [radius=1.5pt];

	\draw [fill, brown] (2,4.2) circle [radius=1.5pt];
    \draw [fill, brown] (-2,3.8) circle [radius=1.5pt];
    \draw [fill, brown] (-2,-1.2) circle [radius=1.5pt];
    \draw [fill, brown] (-1.5,-1) circle [radius=1.5pt];
    \draw [fill, brown] (-1.5,3.5) circle [radius=1.5pt];       
    
    \draw [fill, blue] (1.75,1) circle [radius=1.5pt]; 
    \draw [fill, green] (5,1.5) circle [radius=1.5pt]; 
    \draw[{Circle[blue]}-Latex] (2,0.5) -- (4.9,1.47);
    
\end{tikzpicture}
\caption{\textit{Entrega completada satisfactoriamente.}} \label{fig:M1}
\end{figure}

En este estado final del sistema se puede observar que la entrega del paquete ocasiona que el mismo, ilustrado en azul, pase a ser parte del conjunto de los entregados.

\indent Por el contrario, si el paquete no pudo entregarse al comprador el mismo debe de mantenerse asignado al mensajero hasta que este sea devuelto a la bodega, por lo que el sistema mantiene el paquete en ruta, pero cambia la respuesta obtenida en la operación.

\subsubsection{Regresar paquete a bodega.}
Cuando la entrega del paquete al comprador no concluyo de manera satisfactoria, el paquete debe ser llevado a la bodega, se modifica el estado del sistema por el movimiento del paquete. 

\begin{table}[H]
\center
\makegapedcells
\begin{tabular}{||l|l|l|l||}
\hline
\textbf{Entrada} & \textbf{Caso} & \textbf{Estado} & \textbf{Salida(Tipo : Respuesta)} \\
\hline
\hline
{Paquete} & Exitoso & enBodega & paqueteAlmacenado \\
\hline 
\end{tabular}
\caption{\textit{Regresar paquete a bodega.}} \label{fig:M1}
\end{table}

El resultado de esta operación siempre es \textit{paqueteAlmacenado} que es de tipo \textit{Resultado} y preserva la consistencia en el estado del sistema permitiendo que el paquete pueda ser procesado de nuevo, por ejemplo se puede intentar la entrega del paquete en los días subsiguientes. El esquema que representa la operación es el siguiente:

\begin{schema}{RegresarBodega}
\Delta EntregaPaquete\\
p? : Paquete\\
m? : Mensajero\\
r! : Respuesta
\where
p? \notin enBodega\\
p? \in  enRuta\\
m? \in \ran asignado\\
enRuta' = enRuta \setminus \{p?\}\\
asignado' = asignado \setminus \{p? \mapsto m?\}\\
enBodega' = enBodega \cup \{p?\}\\
pertenece' = pertenece\\
entregado' = entregado
\end{schema}

Este esquema representa el otro posible estado final del sistema. La representación gráfica del mismo se muestra a continuación. 

\begin{figure}[H]
\centering
\begin{tikzpicture}

    \node at (6.72,4.75) {Paquete};

    \draw  (-2.5,-1.5)  rectangle (7.5,4.5);

    \draw[fill=gray!30] (0.7,1.5) ellipse (3cm and 2.5cm);
    \draw[fill=white] (1.8,1) ellipse (0.75cm and 1.25cm);
    \draw[fill=white] (-0.3,1) ellipse (1cm and 1.5cm);
    
    \draw[] (5.7,1.5) ellipse (1.5cm and 2cm);    
    
    \node at (5.7,2.5) {entregado};
    \node at (0.5,3) {pendienteEntrega};
    \node at (-0.27,1.3) {enBodega};
    \node at (1.8,1.5) {enRuta};
    
    \draw [fill] (0.2,2) circle [radius=1.5pt];
    \draw [fill] (-1,0.75) circle [radius=1.5pt];
    \draw [fill] (-0.75,0.5) circle [radius=1.5pt];
    \draw [fill] (-0.5,1) circle [radius=1.5pt];
    \draw [fill] (-0.5,-0.15) circle [radius=1.5pt];

	\draw [fill, brown] (2,4.2) circle [radius=1.5pt];
    \draw [fill, brown] (-2,3.8) circle [radius=1.5pt];
    \draw [fill, brown] (-2,-1.2) circle [radius=1.5pt];
    \draw [fill, brown] (-1.5,-1) circle [radius=1.5pt];
    \draw [fill, brown] (-1.5,3.5) circle [radius=1.5pt];       
    
    \draw [fill, red] (0,0.5) circle [radius=1.5pt]; 
    \draw[{Circle[blue]}-Latex] (2,0.5) -- (0.1,0.5);
    
\end{tikzpicture}
\caption{\textit{Regresar paquete a bodega.}} \label{fig:M1}
\end{figure}

Como se aprecia en la Figura 6, el cambio en el sistema, a nivel de los paquetes, consiste en que el paquete pasa de estar \textit{enRuta} a \textit{enBodega}. El otro cambio es en referencia a la asignación del mensajero y los movimientos que esto provoca en los diferentes conjuntos de paquetes.

%\newpage
%\section{Resumen de entradas, salidas y respuestas.}
%En esta sección se presenta un resumen de las operaciones del sistema, sus entradas, el subcaso que cubre cada una y el cambio que provocan, así como las respuestas para cada escenario.
%
%\subsection{Cuando la entrada es un paquete.}
%El cuadro que se presenta a continuación resume el funcionamiento del sistema para cuando la entrada es de tipo \textit{Paquete}.
%
%
%
%\subsection{Cuando la entrada es un mensajero}
%En el caso de los mensajeros, como consecuencia de la asignación de los paquetes pueden presentar dos estados a saber: \textit{asignado o disponible}. Este trabajo está centrado en el flujo que describe el movimiento de los paquetes por lo que el tema de la asignación de los mensajeros está definido en función de los paquetes y se manejan como respuestas (ver sección 3.3).

\newpage
\section{Trabajos futuros.}
Tanto en el resumen inicial como en la definición del problema se establece que el alcance de este trabajo se limita a un subproceso que forma parte de un sistema más complejo que envuelve todos los aspectos necesarios para que una compra realizada en el exterior llegue de forma adecuada al comprador.\\[\baselineskip]
\indent Las posibilidades de trabajos futuros referentes a este tema se pueden abordar tanto desde mejoras a lo que acá se ha presentado, así como, en modelar y formalizar los subprocesos que, fueron mencionados brevemente en el resumen inicial y que anteceden o preceden al flujo desarrollado por los autores de este trabajo.


\newpage
\begin{thebibliography}{00}
\bibitem{b0} Software Development with Z, J.B. Wordsworth, IBM UK Labs Ltd. First Printed 1992
\bibitem{b1} The Z notation: A reference Manual, J.M . Spivey, 2nd Edition, 1998.
\bibitem{b2} Ingeniería de Software, Referencias varias, Prof.Ignacio Trejos Zelaya, ITCR.2014. \url{https://tecdigital.tec.ac.cr/dotlrn/classes/MC/MC7204/S-2-2018.SJ.MC7204.40/file-storage/#/58119055}
\bibitem{b3} Fuzz Manual, J.M. Spivey, 2nd Edition, Oxford, England, 2000.
\bibitem{b4} A guide to the zed style option, Spivey, Mike, 1990.
\bibitem{b5} Formal Specification Using Z: A Modelling Approach, L. Bottaci J. Jones, 1996.


\end{thebibliography}

\end{document}

